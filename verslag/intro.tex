\section{Introduction}
\label{sec:intro}


\npar This paper will outline a \textit{Delegate Multi-Agents System (DMAS)}
solution that uses \textit{Ant Colony Optimalisation (ACO)} techniques to solve
the dynamic \textit{Pickup and Delivery Problem (PDP)}. The performance of this
solution will be compared with two other classic \textit{Multi Agent Systems
(MAS)}, \textit{ContractNet} and \textit{Gradient Field}.

\npar The considered PDP assumes that a truck can only transport one package at
the time and that new packages can randomly be added to the system. The
infrastructure will also remain the same and congestion on road segments will
not occur.

\npar All of the agents in the presented DMAS solution are simulated in a virtual environment (e.g. a private cloud). It is not very affordable or even practical to equip every
pickup and delivery location (in the physical environment) with a communication device. These locations can change a lot and these devices are necessary because agents on these locations have to be able to communice with each other.
Because the whole DMAS is run on the virtual environment, truck drivers (in the physical environment) will only need some kind of GPS device to receive the planned path from the simulation in the virtual environment.

\npar The purpose of this paper is only to outline the implementation of an ACO based DMAS for PDP. Therefore, the reader is supposed to have a decent knowledge of \textit{Multi-Agent
Systems (MAS)} and the applications of \textit{Ant Colony Optimalisation (ACO)}
techniques in \textit{Delegate Multi-Agent Systems (DMAS)}.

